\documentclass[../main.tex]{subfiles}

\begin{document}

\par Cloud Storage attracts a lot of interests but unfortunately, the cloud's benefits come at a cost of some security issues (e.g: data theft, data leaks, password leaks \cite{wiki:dropboxcritisism}). On top of security, users privacy is a subject that is becoming a priority to most people. Among Cloud Technologies, we can distinguish Cloud Storage (e.g: Amazon S3) used by application developers and Personal Clouds (Dropbox, One Drive, Google Drive, etc) oriented for the end-user only. It has been proven throughout the years that Personal Clouds companies like the cited ones suffered some security issues and if not, have been highly controversial considering users privacy.
\par With this in mind, it could be interesting to outsource the data to the client without outsourcing the Cloud control as per \cite{10.1145/1655008.1655020}. What this means is that instead of trusting the Personal Cloud to protect our information, we consider it is up to the client to design a protocol that secures his information before giving them to the Personal Cloud. It is similar to the edge-centric concept aborded in \cite{10.1145/2831347.2831354}. In this way, the users keep only the advantages of the Personal Cloud (mostly the sharing capabilities and the remote storage) while discarding the security issues.
\par As said in the beginning, users privacy is important in the world of Personal Cloud. Sadly, papers have proven that securely store personal data under the new 2018 GDPR is extremely difficult and requires grounds up solutions (Cfr. \cite{234843} and \cite{234729}). Furthermore, this also heavily impact real-time compliance. Users needs from storage systems dealing with sensitive personal data to enforce compliance. One of the ideas behind GDPR is that personal data should be seen/used only when it is for a legitimate purpose. GDPR states that users should know how and when their data are being processed. This is something that all Personal Cloud provider lacks, once we share our data with other users we don't know how these users are using them (e.g: once we share a folder with a friend, we might want to know which file in the folder our friend is accessing and for what purpose).  This introduces the concept of auditability that GDPR requires: allowing to trace who access which data and for what purpose.
\par Furthermore, with popular Personal Cloud provider, we lack the opportunity to set a rigorous control policy allowing a user to only view or edit a file. This is quite powerful as we often face a real-life situation where we want peoples to view our work without being able to edit it.
\par Following our security scenario (security outsourced at the client side) sharing capabilities becomes an interesting and non-trivial challenge. Indeed, if we secure our information through encryption, how are we supposed to securely share this information with other users? The problem can be split into two: 
\begin{enumerate}
    \item Sharing the encryption key between users. This problem is trivial with an online server through a PKI (Public Key Infrastructure) and the HTTPS protocol. This problem becomes way more difficult when the exchange is between two clients without the aid of external third-party servers.
    \item Revocation is extremely problematic with this approach. Indeed, as the encryption key has been previously shared with allowed users, the owner has no other choice but to change the encryption key (as all the allowed users know the encryption key). This also means that the owner must furthermore re-encrypt all the shared information.
\end{enumerate}
\par An ideal solution for this problem would be that the client never knows the encryption key (the client software never reveals the key to the client). Sadly, no widespread technologies currently exist to enable that.

\medbreak
\par To answer the challenge of edge-controlled Personal Clouds in the age of the GDPR, we propose LAUXUS: an auditable and secure Personal Storage. This software is run on the client's computer and acts as a transparent layer encrypting information on the fly before uploading them on the Personal Cloud. As stated above, GDPR compliance and edge control are difficult to achieve in systems where we can have malevolent users or malware installed onto their devices (from this perspective, the cloud is more secure than the end-user device). To cope with this issue, we chose to leverage secure computation on the end-user client thanks to Intel SGX Enclaves. Furthermore, LAUXUS has the property to only trusts himself (and instances of himself) and shares the storage content only with other instance of itself. On top of that, information handled by LAUXUS never leaves the program itself. The above two functionalities are also a motivation to leverage SGX Enclave.
\par This paper contributes by developing and designing a protocol respecting the above restrictions. Plus, it also acts as an interesting use case of SGX Enclave in the hope of making Enclave more widespread.
\par At the time of choosing this subject, no similar work had already been done. In the meantime, a paper wrote by J. Djoko called NeXuS\cite{djoko2019nexus} was published. Although we asked to cooperate on their code (on which they worked nearly 3 years), they preferred to continue working on their own and making the code public one year later (way too late for us to work on it). Therefore, we based our model and protocol on his and re-implement what he has done, which is in itself a challenging work.  Furthermore, we went far beyond the objectives of NeXuS with the innovative auditability aspect (which was never discussed inside NeXuS nor in the modern state of the art). Plus, this work also serves as a vulgarisation paper to the NeXuS paper which is very short and extremely technical (from a non-expert), in the hope of helping the growth of SGX Enclaves outside the specialist area. In a nutshell, this work brings life to the first Personal Cloud supporting edge-control (as NeXuS did) but that also enforces auditability and purpose-aware access.

\end{document}