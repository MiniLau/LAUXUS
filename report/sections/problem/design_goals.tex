\documentclass[../main.tex]{subfiles}

\begin{document}

\section{Design Goals}
\label{section:problem:design_goals}

\par After reading the previous sections, we can now establish the different goals that we wish to reach when building our solution.
\par \textbf{Practicality}: First of all, we want our solution to be transparent to the users. Their typical business workflow should not be affected by the solution. In the same manner, the solution should only slightly impact performance. Data Management should be fast and easy, at the reach of anyone.  This data management refers to allowing a user to access a resource and revoking a user from this resource. Here, we are not considering the complexity of the installation procedure as it is not relevant.
\par \textbf{Strictly-enforced access policy}: The solution should also maintain a strict and secure access policy. Only allowed users can access the filesystem and an entitlement policy must be put in place to authorise access to each directory and file. By entitlement, we mean the possibility to either read, write or execute the resource (or any combination of these). The importance of these access policies forces them to be encrypted and tamper-evident. This access policy is part of what we aforementioned called "Data Management".
\par \textbf{Portability}: The usability of this filesystem should not be limited to a single computer. A user must be able to access a filesystem and use it independently of the computer he is using, as long as it has been correctly setup. This means that a user must have some sort of credentials to prove its identity to the system. The end-user can choose to store these credentials in many different ways that we will not discuss here. However, the form and the authentication procedure will be a strong focus in the next chapter.s
\par \textbf{Confidentiality}: Any unauthorised entities will not be able to gain accurate information about the users or the files contained inside the filesystem. For example, the patient's name can be visible inside directories or files name which is the pieces of information we don't want to disclose (e.g: naming a file "\textit{john-tumor.pdf}" divulge personal information that could be used by malicious users). However, we are not concerned about disclosing the structure of the filesystem hierarchy (e.g: knowing that under the root directory there are 10 files and 3 folders). This confidentiality goal implies that our cryptographic solution (the algorithm used) must be strong. 
\par \textbf{Purpose-aware access and write-only auditing files}: Every action took by an authorised user, may it be reading/editing, must be recorded along with a justification, before being able to proceed with the action. In practice, this means that when Bob wants to save his Microsoft Word document to disk, a prompt message should appear asking him why these modifications were needed (e.g: update of the medical record, new insights about a patient, etc). These justifications must be secured and available to adequate parties immediately. Needless to say that these auditing files must be tamper-evident and no one, except the adequate parties, can read them. Even adequate parties can't change them.
\par \textbf{Security}: The solution should not trust the user together with its device (at the exception of the SGX Enclave in the device's CPU). The reason for not trusting the users is to protect the Filesystem against users who wish to read or edit files they are not supposed to access. Not trusting the user's device means that even if an attacker has access to the computer (e.g: through some kind of malware of physical access), he can't retrieve any sensitive information about the Filesystem.

\end{document} 