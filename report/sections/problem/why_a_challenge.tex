\documentclass[../main.tex]{subfiles}

\begin{document}

\section{Why is it a challenge ?}
\label{section:problem:why_a_challenge}

\par In this section, we are going to analyse the problem according to what is currently available in the state of the art. To do so, we will look at a naive solution and demonstrate its limits (and thus prove the limits of current technologies).
\par The solution could be to simply encrypt, with a secure key, the filesystem and then upload it to the remote storage. When the owner wants to share this with someone else, he can simply give him the encryption key and the intended user can securely decrypt the content. This can easily be scaled up to any number of users to share with. This solution has limitations:
\par \textbf{Access policy} becomes impossible using this approach. Once the owner transmits the encryption key to a user, this user can decrypt the entire filesystem. No granularity can be made by the owner. A possible workaround for creating this granularity is to encrypt each file with a unique different key. Thus transmitting then multiple keys when the filesystem is shared.
\par Beyond the scalability issues, this leads us to the \textbf{user revocation} problem. Indeed, the only way for the owner to revoke users access is to generate a new key for each file the user had access to. The owner must then re-encrypt the entirety of the concerned files and transmits all the newly generated keys to the remaining authorised users. This means that the revocation procedure has a huge overhead, scaling up proportionally with the number of files the revoked user has access along with these files size.
\par Most of all, \textbf{encryption keys protection} is impossible in this situation. All the authorised users know each of the keys they have been given and it is up to them to chose whether or not they want to share this key with others.
\par Lastly \textbf{transparency} can only be achieved by using a layer intercepting IO system calls (which means that the user's device must be trusted to deliver the correct data to our software). In this way, the approach works no matter the user-space application the end-user is using. This layer allows us to manipulate the data saved to disk however we want. Also, this layer is the core mechanism of our solution allowing to make all required operations for our filesystem to be at the same time confidential and auditable.

\end{document} 