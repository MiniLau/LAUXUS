\documentclass[../main.tex]{subfiles}

\begin{document}

\section{\textit{FUSE}}
\label{section:theoric:fuse}

\par \textit{FUSE} stands for Filesystem in User Space. It is used to override the typical behaviour of the UNIX Filesystem. By its ease of use, \textit{FUSE} allows to easily create personalised filesystem without modifying the OS. It is especially used in the research world to develop and prove innovative filesystem. Some papers have already been published with the idea to protect a filesystem from the cloud such as SGX-FS\cite{8590996} and SafeFS\cite{10.1145/3078468.3078480}. But it is not limited to creating a secure filesystem as per \cite{fuseExamples}.
\par \textit{FUSE} consist of a Kernel module and a user-space library. It works by forwarding the system calls from the kernel into the user-space library for custom behaviour. A \textit{FUSE} instance is launched by mounting it to a given directory. This directory will then mirror the behaviour described inside the user-space library. A great advantage of using \textit{FUSE} is that it can be used as layers on top of each other (e.g: combine a filesystem that turns the path in lower case and one that creates a directory when the filename is composed of a "-"). Another often favoured advantage is that it is very easy for a lambda developer to create a custom Filesystem thanks to its user-friendly user-space library.
\par Lastly, we can note that there are two versions of the user-space library: either a low level one or a high level one. The main difference is that the first one works with inode whereas the latter one works with filenames (and performances are slightly different in the favour of the low-level API).

\medbreak
\par In a nutshell, the advantage of \textit{FUSE} is that it provides transparency and that a \textit{FUSE} instance doesn't impact the other \textit{FUSE} instances. Its main drawback is its performance: \cite{vangoor2017fuse} and \cite{10.1145/1774088.1774130} found out that optimised \textit{FUSE} can perform within 5\% of native filesystem. However, every custom file-system might not be friendly with \textit{FUSE} which degrades drastically its performance (up to 83\%). Furthermore, it has been proven that \textit{FUSE} file-systems are CPU intensive (an increase of 31\%). However, this performance drawback is not important as this work is more focused on security than performance. In our case, performance is acceptable even if it is 3 times slower than standard filesystems.

\end{document}