\documentclass[../main.tex]{subfiles}

\begin{document}


\section{Limitations}
\label{section:analysis:limitations}
\par Now that we have cover the performance of LAUXUS, we can look at its limitations. As we have already cover the security limitations in the previous section, we will focus more on the functionalities limitations.
\begin{itemize}
    \item \textbf{File size:} There is a maximum size to the files that LAUXUS can manage. This limitation is caused by the fact that we load the whole Filenode metadata structure in memory (more precisely on the Heap). As the file becomes larger, it is composed of more block and thus of more block keys stored inside the metadata structure\footnote{20 MB File with 4KB block -> 5000 block keys of 32B -> metadata structure of 160KB -> Heap Size = 0x27100}. Of course, this problem can be temporally solved by either increasing the Heap size or increasing the block size (which is by default to 4KB). A more permanent solution would be to use pagination on the block keys of the metadata structure.
    \item \textbf{UUID generation:} There may be collision when generating the UUIDs for the users or the files. Indeed, all the UUIDs are randomly generated. However, each UUID is composed of 16 random element where an element has 16 possibilities. This means that, following the birthday paradox, there is a 50\% chance of having a collision after generating $16^8$ UUIDs. Theoretically, it may not be the best idea but in practice, we will never encounter a collision as there are just too many permutations.
    \item \textbf{Prompt action purpose:} Currently, there is no nice GUI to ask the user to provide the purpose of his action. Despite huge research, I wasn't able to find a way to ask for the user purpose. As it was not the primary goal of this work (the goal was more to prove that the concept was feasible) I chose to let the purpose to be a static string. Once someone finds out how to ask for a user input from LAUXUS, everything will be good to go as the back bone of handling user purpose is already implemented.
    \item \textbf{Syscall not implemented:} We didn't implemented all the \textit{FUSE} system calls because most of them were out of the scope of our work. As the default behaviour of \textit{FUSE} is to through an error code when a call is not implemented (instead of acting as a pass-through), some program may not work (such as Git, etc).
    \item \textbf{Persistence of the user entitlement:} Currently, when a user is revoked from the Filesystem, only the Supernode user database is updated, not the user entitlement of the files. We chose to do it this way because if we had to load and update all the nodes of the Filesystem when a user is revoked, we would incur a huge overhead to the revocation procedure which should be fast. As the user entitlement works with the user UUID, a new user might inherit the user entitlement of the previous user with the same UUID. However, as discussed above, this probability of a UUID collision is extremely slim thus don't incur an important security threat. A way to decrease even further this "threat", we can write a small script that he is run before mounting LAUXUS that check the Filesystem structure and along the way correct the small incoherence (such as the when there is the UUID of a revoked user inside a file user entitlement).
    \item \textbf{Concurrent writing operation:} Currently, LAUXUS will end up in an uncorrect state when there are concurrent updates on the same file (e.g: two users edit the same filesystem when offline and then upload them on the remote storage). Indeed, as we must update the IV of the metadata structure on each file update (cfr. Section \ref{section:lauxus:metadata_encryption}), if two user edit a file at the same, it will result in two different IV and classical merging algorithm won't be able to resolve this. The simple solution to this is to use AES GCM SIV instead of AES GCM due to its nonce re-usability (cfr. Section \ref{section:lauxus:metadata_encryption}).
\end{itemize}


\end{document}